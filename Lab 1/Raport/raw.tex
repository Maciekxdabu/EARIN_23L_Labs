\documentclass{article}

\title{EARIN Lab 1}
\author{Maciej Domański 303853 \\ Gabriel Skowron-Rodriguez 300364}


\begin{document}
\maketitle

% NEW SECTION
\section{Intro}

On this first laboratory our task is to implement the 2D maze-solving algorithm \textbf{Greedy best-first}.
Our maze will be loaded from a file and defined as:
\begin{itemize}
    \item[S] - Starting position
    \item[E] - End position
    \item[O] - Empty space
    \item[X] - Wall
\end{itemize}

We will provide a visualization of the working algorithm and test it for two different heuristic functions to compare results.

% NEW SECTION
\section{Algorithm}

Maze-solving algorithms in general are going cell by cell from starting position in a specific pattern (algorithm) in order to find their way to the end position.
Those algorithms determine the next step using a heuristic function.

Such function returns a value for each cell in the maze (usually the distance from the End position calculated in a specific way).
Then depending on the algorithm next cell or cells are chosen and the algorithm begins anew.

In our specific case i.e. \textbf{Greedy best-first} algorithm, at each step we look for the smallest result of the heuristic function adjacent to currently analyzed cell.
So basically we look for a cell that is closest to the end position regardless of the maze layout.
In the next step we will look again at the ones adjacent to the cell we chose before.
The process is repeat until the end cell is reached.


% NEW SECTION
\section{Our program}


% NEW SECTION
\section{Summary}

\end{document}